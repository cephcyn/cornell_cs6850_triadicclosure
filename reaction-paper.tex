  The models presented in class are simple and beautiful--particularly the idea of modeling collective participation as a function of the number of other folks participating. However, human behavior is messy and complicated and finding the right levels of abstraction to capture enough of the complications that make the problem of interest uniquely difficult to model with existing methods is a challenge. We hope to model participation on a dynamic graph where individual nodes have an unreliable awareness of their neighbors’ actions, and present several project proposals centered around this idea, as well as discuss a few relevant background works.

\section{Proposed Projects}


\textbf{Proposal 1: }To extend the idea of weighted social networks contributing to the idea of threshold models: What if we weight connections between nodes differently to prioritize long-range edges over short-range ones? 

A motivating example is the slow shift of political opinion. As a relatively taboo and contentious topic, discussion of politics among friends and family (the traditional links of  the social graph) can be less frequent than the exposure individuals have to public political discourse, such as through the news or social media. We propose modelling individuals (and sometimes organizations) as individual nodes. Individuals are connected according to their social connections and also where they receive their news and political arguments (people they follow on Twitter, news outlets such as the NYT, or through organizations they are a part of, such as their church, volunteer group, or weekly dance class). Individuals have infrequent conversations with their close connections, but receive reliable broadcasted messages from public figures. Unlike most networks where nodes most reliably see the actions of their close neighbors, in our model, nodes most reliably see actions of nodes that are still connected to them, but might not be in a densely connected cluster. \\

\textbf{Proposal 2: }Instead of assuming universal awareness that some new cascading idea exists, we could model the spread of idea awareness and perceived popularity together with the spread of idea adoption.

We model individuals as nodes in a graph and the social connections as edges between nodes. Each has a threshold for participation with a group $T$. $T$ is a function of the number of neighbors the node is aware of in each group, $T(A(n_i))$, where $n_i$ is the set of neighbors of node $i$, and $A(n_i)$ is the number of neighbors of node $i$ that $i$ can observe has chosen to be in group. In this case, if $A(n_i)$ is lower than a particular threshold, node $i$ would not participate. However, if $A(n_i)$ is very high, some nodes may also not want to participate (too popular). There’s some sweet spot of awareness of neighbors' actions where node $i$ would be most likely to participate in action A.\\

\textbf{Proposal 3: }Previous work on small-world graphs casts long-range connections as a source of novelty (people meet new people through their long-range connections). As people explore the new found social-circle, their connections to the new circle become strengthened and the group might become denser. For how long does a connection stay long-range? We propose exploring how weighted triadic closure (adding an edge between node A and C if they both have connections to node B) affects the rate a process spreads across a network. We increase the probability that an edge gets added as the number of nodes in common between node A and C increases. 

Additionally, we will explore how too much information can dampen information spread. For instance, a researcher choosing an area to work in might want a sweet spot of an area that has enough interest to sustain collaborations, but not one that’s so popular that it would become crowded. Thus, in addition to each node having a threshold on the number of neighbors who have joined for the given node to join, each node will have a threshold at which they leave. As nodes become more aware of more agents, they might discover the field is more crowded and leave.

Another variant on this would be to use a threshold on the proportion of neighbors that have joined. A design decision that we will additionally explore here is whether to interweave phases of triadic closure with the spread of information (aka, the graph is changing while information spreads), or whether to modify a graph then evaluate the spread of closure (aka, the graph changes first, then information spreads). It’s much easier to reason about the graph changing first than if the graph were changing while information was spreading, but we could potentially observe interesting dynamics emerge where triadic closures could slow the spread of information.

\section{Literature Review}

Our proposed models build on large bodies of work extending “Threshold Models for Collective Behavior.” We will give a survey of a few selected works and then discuss possible extensions of the work and how these works inform our project development. 

\cite{granovetter} introduces a simple model for estimating how many people would partake in collective action as a function of how many people in the population have already joined. This paper made several simplifying assumptions that later work then explored. For example: they assume that as more people participate the more likely the rest of the population is to join as well. Later work explored scenarios where a good is less valuable if too much of the population has adopted it such as the El Faro Bar problem. 

The paper poses the question of how social structure changes participation and then laments that this cannot be easily studied analytically. This then kicked off a body of work extending threshold models to graphs where whether an individual participated depended on only the actions of its neighbors. The original work assumed that everyone in the population could observe the behavior of everyone else. Since then, work surveyed in the book \cite{networks-crowds-markets} ch 19 has adjusted this “common knowledge” in several ways: limiting an individual’s observations to its neighbors on a graph, allowing observations to decay through time, and allowing individuals to reason about what their neighbors would do given the information they have. Additional extensions could include making an individual’s threshold a function of the distance between the individual and the neighbors that are participating, and or adjusting the reliability of information received from neighbors.

A fascinating result from this model is how two populations where the vast majority of individuals across the populations have the same thresholds can have dramatically different behavior due to smaller fluctuations in the number of individuals with lower thresholds (or also how small the thresholds are on the low end). This is because individuals with lower thresholds kickstart the cascade, and not enough individuals at the lower range means a shorter cascade. 

\cite{immorlica} builds on top of \cite{granovetter} by focusing on the dynamics between two or more competing actions in a social network. They focus on the example of competing tech products. If the environment is initially filled with users of a lower-payoff product B, one node brings in higher-payoff product A, and from then on each node may choose to adopt only A, only B, or both A and B with some compatibility cost (as it takes more effort to maintain both together) based on the payoff it will get with its connected nodes, how would the popularity of each product change over time?

They find that there is a complex boundary curve for the relationship between compatibility cost and graph structure which distinguishes whether A will extinguish B, v.s. both continuing to coexist. In the context of their motivating example: if an old product is too compatible (or not compatible enough) with a newer, better-performing one, it will eventually die out when the newer product gets released.

They also extend the idea to more than two competing products: if there are two old products that currently coexist and a single new one comes along, increased compatibility between the two older products means that the new one is unable to dominate the network.

This paper builds on the original threshold model \cite{granovetter} by focusing on incorporating network structure and considering the possibility of at least two competing yet non-exclusive movements that nodes could potentially adopt. Instead of only a single activity (choosing to join a riot in a plaza), there are multiple possible activities, each with its own set of unique advantages, and the acknowledgement that doing multiple activities at once can be uniquely tiring.

However, while it visualizes the impacts of varying compatibility costs and network degrees on a range of theoretically interesting infinite graph structures, one thing it doesn't do is test the concept out on real-world social networks, which can still be a good step away. It would also be interesting to retroactively attempt to identify a heuristic compatibility cost based on the history of multiple competing products or ideas, to see how well this model may reflect actual interaction patterns.

Furthermore, while it briefly investigates more-than-2-product interactions, it neither examines the potential of having distinct compatibility costs between each pair of products, nor does it consider the potential changes in compatibility between different products that their developers or users may make over time to try to protect their existence. Understandably, it is difficult to explore more-than-2 product interactions, but simulations of competing objectives could be useful as it more closely mirrors the real world where people are limited in their resources and need to choose how to spend their time and energy on possible products. 

There is additional work that explores the spread from multiple sources but with a single product. This work often asks the question “which nodes are most influential”, and thus, where in the graph should initial adopters (seed nodes) be. \cite{kempe} and \cite{chenEfficient} introduce algorithms to find these seed nodes based on features of the graph.

\cite{salehi} surveys multiple works centered around multilayer network threshold models, which extend the idea of regular (monoplex) network-based threshold models to include multiple potentially overlaid social networks and how they may impact the spread of information. One common motivation for multilayer networks is the existence of multiple online social platforms: not every person is present on every single social platform, so only considering one platform isn't enough. Furthermore, there may be different methods of communicating across each platform and there may be varying costs for someone who received information on one platform to send it out on another platform.

In these multilayer networks, it's typically considered possible for a cascade (joining a movement, adopting a product, learning an idea) to spread between nodes on the same "layer" graph, across layers between nodes that identify the same individual, or sometimes spread across layers between nodes that represent different individuals. These different types of links may have different transmission costs and probabilities, to reflect their different natures.

In the survey, they outline a range of different metrics that have been used in cascade behavior studies, such as infection rate and transmissibility which both relate to the relationship and probability of infection spreading between two connected nodes, epidemic threshold and outbreak probability which are the minimum transmissibility required for and the overall probability of a given seed infection causing an epidemic respectively, and survival probability being the probability of a particular infection continuing to exist at any given time.

They raise several open questions: how do you get an accurate representation of graph structure if you may have hidden nodes in a particular graph or hidden layers, and how do you define structure metrics such as degree for multilayer networks? Furthermore, we wondered how the behavior of cascades on a multilayer network could be somehow generalizable to their behavior on an analogous monoplex (single layer) network, as well as whether simulating multiple cascades with different collaboratory purposes would make sense across different layers (for example: awareness spread vs. actually joining in on an action (proposal 2)).

While the multilayer network is a useful image for modeling different methods of communication, we wondered if it might be just as easy to have one graph that contained all the nodes then label each node and edge as different categories. For instance, many social media platforms could be viewed as a graph where switching between edges representing different platforms incurred an additional cost. 

\cite{centola} presents a lighter weight version of multilayer graphs that explores the spread of two different ideas across the same graph. The spread of the two ideas are intertwined but every node has different thresholds for them. We proposed extending this work in proposal 2 by incorporating the multiple thresholds (join vs. leave) from the El Faro Bar problem. 

\section{Discussion}
The simplifications made in each of these papers invites additional work studying to what extent these simplifications are necessary or robust for the phenomena discussed to arise. Indeed much work has been written exploring these assumptions (more than we were surely able to find, read, and discuss). Each of the papers presented could be complicated by turning the knob on some of the following assumptions: It’s better to be part of the group; more information is better; graphs are static; thresholds are static; direct neighbors are more trustworthy than further nodes; there is one node competing across the graph; information is trustworthy.

The list continues to go on, but for our work, we'll start with one of our proposed projects, then possibly vary the assumptions mentioned here and observe how that affects the percentage of nodes the network can spread to and how many iterations it takes to reach its maximum spread.

% In class, we saw a few examples where several questions could reduce down to the same question. Something similar to this could be modeling how benefits might decrease as more nodes participate as an increase in the threshold to participation. 


\printbibliography